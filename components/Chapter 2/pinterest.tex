\par 

\subsection{Pinterest}
Pinterest là một xã hội được thiết kế để quản lý và chia sẻ ý tưởng về hình ảnh, ra mắt vào năm 2009. 
Hình ảnh được chia sẻ thông qua thao tác được gọi là pin (ghim).\par \par 


\textbf{Điểm nổi bật của Pinterest:}
\begin{itemize}
    \item Home Feed có thuật toán tìm ảnh tương đương.
    \item Thuật toán gợi ý chủ đề theo theo từ khóa tìm kiếm.
    \item Tính năng “Pin”: “Pin” về cơ bản có nghĩa là bạn lưu trữ một bộ sưu tập hình ảnh với nhau để tạo thành một nhóm. Mỗi hình ảnh riêng lẻ là một pin mà người khác đã đưa lên. Tính năng Pin áp dụng cho tất cả ảnh xuất hiện ở Home Feed hoặc mục tìm kiếm.
    \item Tính năng follow tác giả.
    \item Bảng: tính năng lưu trữ và chia sẻ tập hợp tất cả các Pin. các Pin công khai được hiển thị trên trang Profile.
\end{itemize}

\subsubsection{Ưu điểm và nhược điểm}
    \subsubsubsection{Ưu điểm}
        \begin{itemize}
            \item Pinterest rất dễ sử dụng, quá trình đăng nhập cũng không mất nhiều thời gian.
            \item Đây là kênh mạng xã hội có khả năng lưu giữ một lượng lớn hình ảnh cho người dùng.
            \item Pinterest hữu ích để quảng bá sản phẩm và kinh doanh online.
            \item Theo thống kế, Pinterest có tới hơn 81\% người dùng là nữ giới nên là môi trường tiềm năng để tiếp thị các sản phẩm làm đẹp.
        \end{itemize}

    \subsubsubsection{Nhược điểm}
        \begin{itemize}
            \item Vì là mạng xã hội chủ yếu lưu giữ hình ảnh nên Pinterest hoàn toàn là các hình ảnh theo chủ đề người dùng yêu thích.
            \item Hình ảnh của người dùng không được bảo vệ bản quyền vì không có công cụ hỗ trợ.
        \end{itemize}

