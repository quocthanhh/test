\par 

\subsection{YouPic}
YouPic được thành lập vào năm 2012 bởi Giám đốc điều hành hiện tại là Navi Razazi. 
Công ty có trụ sở chính tại Gothenburg, là thành phố lớn thứ hai ở Thụy Điển. 
YouPic có hơn ba triệu người dùng trên khắp thế giới.\par

Về bản chất, nền tảng này hoạt động giống như sự giao thoa giữa Twitter và Instagram. 
Ngoài tính chất chia sẻ hình ảnh, người dùng có thể chia sẻ lại và like những bức hình mình yêu thích.\par

Nền tảng này được thiết kế để giúp các nhiếp ảnh gia không chỉ chia sẻ tác phẩm của họ mà còn nhận được phản hồi. 
YouPic cũng cho phép người dùng được thuê cho công việc của khách hàng, trong khi quyền hình ảnh vẫn thuộc 
về nhiếp ảnh gia sau khi xuất bản.\par

Youpic không chỉ là một nền tảng truyền thông xã hội dành cho các nhiếp ảnh gia mà nó còn có rất nhiều 
khía cạnh truyền thông xã hội đa dạng khác. Giống như các đối tác của Youpic, bạn có thể sử dụng nó để trình
 bày hình ảnh mà mọi người có thể chia sẻ và tương tác.\par

Điều khiến Youpic trở nên khác biệt với các trang web thông thường là nó phục vụ chủ yếu cho các chuyên
 gia hơn là nhiếp ảnh gia nghiệp dư. Nó đóng vai trò như một đường dẫn trong việc trao đổi ý tưởng với các 
 nhiếp ảnh gia và thậm chí cả những khách hàng trong tương lai.\par

Trang mạng xã hội này chứa đầy những thứ mà ngay cả bản thân các bức ảnh cũng bao gồm thông tin kỹ thuật như 
dữ liệu EXIF và thẻ địa lý.\par

Ngoài việc chia sẻ ảnh, nó cho phép bạn truy cập vào các hướng dẫn từ những người giỏi nhất trong lĩnh vực kinh 
doanh. Ngoài ra, bạn cũng có thể kiếm tiền từ ứng dụng bằng cách bán công việc hoặc dịch vụ của mình.\par


\subsubsection{Ưu điểm và nhược điểm}
\subsubsubsection{Ưu điểm}
\subsubsubsection{Nhược điểm}